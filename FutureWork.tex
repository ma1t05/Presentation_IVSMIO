Conclusions

Conclusions are the last section people read in your paper, and therefore it’s what they leave remembering. You need to make sure they walk away thinking about your paper just the way you want them to.

Your conclusions needs to do three main things:

    Recap what you did. In about one paragraph recap what your research question was and how you tackled it.
    Highlight the big accomplishments. Spend another paragraph explaining the highlights of your results. These are the main results you want the reader to remember after they put down the paper, so ignore any small details.
    Conclude. Finally, finish off with a sentence or two that wraps up your paper. I find this can often be the hardest part to write. You want the paper to feel finished after they read these. One way to do this, is to try and tie your research to the “real world.” Can you somehow relate how your research is important outside of academia? Or, if your results leave you with a big question, finish with that. Put it out there for the reader to think about to.
    Optional Before you conclude, if you don’t have a future work section, put in a paragraph detailing the questions you think arise from the work and where you think researchers need to be looking next.

%Things to not do in your conclusion:

%   Introduce new information. The conclusion is for wrapping up everything you’ve done. It’s not a place to say “oh yeah, and we also got result y.” All results should be first presented and detailed in the result section. Think of the conclusion as a place to reflect on what you’ve already said earlier in the paper.
%   Directly re-quote anything you’ve already written. I’ve seen conclusions that are almost identical to the abstract or a collection of sentences from throughout the paper. As a reader, it makes me think the author was lazy and couldn’t be bothered to actually summarize their results for the paper. Take the time to write a proper conclusion so that the reader walks away with good thoughts about your work.
%   Write a conclusion longer than your introduction. A conclusion should be short, and to the point. You’ll rarely see them over 3 paragraphs, and three is often long. A lot of the time they are usually only one or two. Think about a conclusion as a chance to see how concisely you can summarize your entire research project. It’s your “30 second” research spiel.

\section{Future Work}
\begin{frame}{Future Work}
  \begin{itemize}
  \item Develop a simulator to evaluate the quality of the solutions
  \item Design and implement heuristic methods to solve larger instances
  \item Generate instances from real data
  \end{itemize}
\end{frame}
