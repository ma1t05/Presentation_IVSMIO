
\subsection{Background}
\begin{frame}[allowframebreaks]
  Richard Larson (1974) \cite{larson1974hypercube,larson1975approximating}
  proposes the Hypercube, and A-Hypercube models
  for a queuing approach for locating multiple facilities.

  James P. Jarvis (1985) \cite{jarvis1985approximating} incorporates
  location dependent service times characteristics for the A-Hypercube model,
  developing an approximation model for a spatially distributed queuing system
  under general service time assumptions.

  Berman et al. (1987) \cite{berman1987stochastic}
  formulate the Stochastic Queue p-Median problem (SQpM),
  and propose a heuristic approach for locating cooperative service facilities
  on a network.

  Goldberg et al. (1990) \cite{goldberg1990validating}
  propose a nonlinear integer programming model
  based on general service time approximation
  for spatially distributed queuing systems.

\end{frame}

\subsection{Hypercube queueing model}
\begin{frame}
Solution procedures for probabilistic models that use the hypercube model
with optimization procedures 
for server deployment can be found in the literature, for example
Larson (1979) \cite{larson1978structural}, 
Berman et al. (1985,1987) \cite{berman1985optimal,berman1987stochastic}, 
Saydam and Aytug (2003) \cite{saydam2003accurate}, 
Rajagopalan et al. (2008) \cite{rajagopalan2008multiperiod} and 
Iannoni et al. (2008a, 2008b) \cite{iannoni2008hypercube,iannoni2009optimization},
\end{frame}
