
%Emergency service systems: 
%The use of the hypercube queueing model in the solution of probabilistic location problems

\section{Hypercube queueing model}
\begin{frame}
Given a system configuration, 
the hypercube model is able to evaluate a variety of performance
measures relevant for decision-making, 
either region-wide or for each server or region.

These include 
server workloads, 
mean user response times, 
fraction of dispatches of each server to each region,
among others.
\end{frame}

\begin{frame}
There are some basic assumptions for the application of the hypercube model:
\begin{enumerate}
\item Geographical atoms and arrival processes
\item Servers and service processes
\item Server assignment and fixed-preference dispatching
\end{enumerate}
\end{frame}

\subsection{Calibration process of the mean service times}
\begin{frame}
In this emergency system, 
travel times may represent a considerable part of service times. 
It may be advisable to adjust the service times by means of a calibration process,
which can be performed using a simple iterative procedure.

The procedure consists of 
verifying if there are significant differences among 
the input mean service times and the output mean service times (computed by the hypercube model). 
In this case, 
the hypercube is solved using the computed mean service times as inputs, 
until the differences among input and output values are sufficiently small
\end{frame}

\begin{frame}{Probabilistic location models for planning ESSs}

Solution procedures for probabilistic models that use the hypercube model
Further studies combining the hypercube model with optimization procedures for server
deployment can be found in the literature; 
Larson (1979), 
Berman et al. (1985,1987), 
Saydam and Aytug (2003), 
Rajagopalan et al. (2008) and 
Iannoni et al. (2008a, 2008b).
\end{frame}
