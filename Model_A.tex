% Validating and applying a model for locating emergency medical vehicles in Tucson, AZ
% Goldberg 1990

\section{Models}
\subsection{Model A}

\begin{frame}
  This model is based on Goldberg's model \cite{goldberg1990validating}, and it contains assignment
  variables for all possible orders.

  The following assumptions are considered in the model:
  \begin{itemize}
  \item The probability that an adjuster is busy is $\rho$
    and is unaffected by the state of the system
  \item There is a strict ordering of the bases preferred for each zone
    that does not depend on the current state of the system
  \item All calls are answered by an adjuster originating from its base,
    not en route back to the base
  \item The arrival of calls to the system follows a stationary distribution
  \item The model is presented using a 0-queue assumption.
  \end{itemize}
  
\end{frame}


\begin{frame}
  Parameters:
  \begin{itemize}
  \item $V$ set of demand points, with $|V| = n$
  \item $W$ set of potential site for adjusters/facilities, with $|W| = m$
  \item $p$ number of available adjusters
  \item $K$  set of possible orders $\{1,2\ldots,p\}$
  \item $\rho$ is the utilization of each adjuster
  \item $t_{ij}$ is the expected travel time between point \textit{i} and potential site \textit{j}.
  \item $h_{ij}^{k}$ is the probability that $j$ serves point $i$ given that is the $k$-th preferred.
  \item $S_{ij} = \{r\in W | t_{ir} < t_{ij}\}$
  \end{itemize}
  
  Variables:
  \begin{itemize}
  \item $x_j =
    \begin{cases} 
      1 & \mbox{if an adjuster is at potential site } j \\
      0 & \mbox{otherwise}
    \end{cases}$
  \item $y_{ij}^k =
    \begin{cases} 
      1 & \mbox{if adjuster in } j \mbox{, is the k-th to cover demand point }i \\
      0 & \mbox{otherwise}
  \end{cases}$
  \end{itemize}
\end{frame}

\begin{frame}[allowframebreaks]{}{}

{\small
  \begin{equation}
    \min \, \sum_{j=1}^{m}{\sum_{\ell=1}^{k}{\sum_{i=1}^{n}{h_{ij}^{\ell}t_{ij}y_{ij}^{\ell}}}}
  \end{equation}
}
{\small
  \begin{align}
    \sum_{j \in W}{x_j} & = p               &                                  &\\
    \sum_{j \in W}{y_{ij}^{k}} & = 1        &          i \in V, k &\in K \\
    y_{ij}^{k} & \leq x_j                   &  i \in V,j \in W, k &\in K \\
    \sum_{k = 1}^{p}{y_{ij}^{k}} & \leq x_j &          i \in V, j &\in W \\
    y_{ij}^{k} &\leq \sum_{r\in S_{ij}}{y_{ir}^{k-1}} &  i \in V,j \in W, k &\in K/\{1\} \\
    x_{j} & \in \{0,1\}      &                  j &\in W \nonumber\\
    y_{ij}^{k} & \in \{0,1\} &  i \in V,j \in W,k &\in K \nonumber
  \end{align}
}
\end{frame}

\begin{frame}
  This model was tested with a benchmark instance of 30 points, 
  where $V = W$.
  For values of $p = 1,\ldots,7$ the optimal solution was found in less than one minute.

  For other benchmark instances was impossible to find the solution
  because they have more than 300 points, and the proposed model grows to over one million variables.
%  For some large instances ($n = 50$ , $m = 30$)%
\end{frame}
